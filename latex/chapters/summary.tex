\chapter{Summary}

This study presents three tasks from the application of machine learning: data mining, neural network training and inversion. The presented tasks were performed in Python with the assistance of its third-party libraries. \medskip

Inversion has not received much attention since the rise of machine learning, and Python's machine learning libraries have not contained any inversion methods. In this paper a solution was given for computing the inversion of a function approximated by neural networks.\medskip

The goal is to perform the inversion, for what a sufficient number of neural network model has to be established. The necessary tasks are being tackled in this paper. Data mining was used for extracting information from the data set. Multi-layer perceptron models were built and tested with numerous parameter combinations to find the best fitting combination which can predict the outputs the best. Then the inversion of the single element feedforward neural network was completed by the Williams-Linder-Kindermann inversion.\medskip

The Online News Popularity dataset, which is a publicly available dataset was used during the whole process from data mining to the inversion. The training was running in parallel on the supercomputer with thousands of parameter combinations of the multi-layer perceptron. After thousands of iterations the neural network mean test score has not reached acceptable levels. WLK inversion has been tested on smaller datasets with trained neural network with good results. \medskip

This paper contains applied machine learning research in the field of artificial intelligence with the implementation of inversion. In future plans, this explanation will be sent to Scikit-Learn as a scientific solution of the inversion problem.